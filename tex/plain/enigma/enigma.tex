\newif\ifenigmaisrunningplain
\ifcsname ver@enigma.sty\endcsname\else
  \enigmaisrunningplaintrue
  \input{luatexbase.sty}
  \catcode`\@=11
% \else latex
\fi
\catcode`\_=11 % There’s no reason why this shouldn’t be the case.
\catcode`\!=11
%D Nice tool from luat-ini.mkiv. This really helps with those annoying
%D string separators of Lua’s that clutter the source.
\def\luastringsep{===} % this permits \typefile{self} otherwise nested b/e sep problems
\edef\!!bs{[\luastringsep[}
\edef\!!es{]\luastringsep]}
%D \startdocsection[title=Prerequisites]
%D \startparagraph
%D Package loading and the namespacing issue are commented on in
%D \identifier{enigma.lua}.
%D \stopparagraph
\directlua{
  packagedata = packagedata or { }
  dofile(kpse.find_file\!!bs enigma.lua\!!es)
}

%D \startparagraph
%D First, create somthing like \CONTEXT’s asciimode. We found
%D \texmacro{newluatexcatcodetable} in \identifier{luacode.sty} and it
%D seems to get the job done.
%D \stopparagraph
\newluatexcatcodetable \enigmasetupcatcodes
\bgroup
  \def\escapecatcode      {0}
  \def\begingroupcatcode  {1}
  \def\endgroupcatcode    {2}
  \def\spacecatcode      {10}
  \def\lettercatcode     {11}
  \setluatexcatcodetable\enigmasetupcatcodes {
      \catcode`\^^I = \spacecatcode % tab
      \catcode`\    = \spacecatcode
      \catcode`\{   = \begingroupcatcode
      \catcode`\}   = \endgroupcatcode
      \catcode`\^^L = \lettercatcode    % form feed
      \catcode`\^^M = \lettercatcode    % eol
  }
\egroup
%D \stopdocsection

%D \startdocsection[title=Setups]
%D \startparagraph
%D Once the proper catcodes are in place, the setup macro
%D \texmacro{do_setup_enigma} doesn’t to anything besides passing stuff
%D through to Lua.
%D \stopparagraph
\def\do_setup_enigma#1{%
    \directlua{
      local enigma = packagedata.enigma
      local current_args = enigma.parse_args(\!!bs\detokenize{#1}\!!es)
      enigma.save_raw_args(current_args, \!!bs\current_enigma_id\!!es)
      enigma.new_callback(enigma.new_machine(\!!bs\current_enigma_id\!!es),
                          \!!bs\current_enigma_id\!!es)
    }%
  \egroup%
}

%D The module setup \texmacro{setupenigma} expects key=value, notation.
%D All the logic is at the Lua end, not much to see here …
\def\setupenigma#1{%
  \bgroup
    \edef\current_enigma_id{#1}
    \luatexcatcodetable \enigmasetupcatcodes
    \do_setup_enigma%
}
%D \stopdocsection

%D \startdocsection[title=Encoding Macros]
%D \startparagraph
%D The environment of \texmacro{start<enigmaid>} and
%D \texmacro{stop<enigmaid>} allow enabling of Enigma encoding in
%D different parts of the document.
%D \stopparagraph

\edef\c!pre_linebreak_filter{pre_linebreak_filter}
\def\do_define_enigma#1{%
  \@EA\gdef\csname start\current_enigma_id\endcsname{%
    \endgraf
    \bgroup%
    \directlua{%
      if packagedata.enigma and packagedata.enigma.machines[ \!!bs#1\!!es] then
        luatexbase.add_to_callback(\!!bs\c!pre_linebreak_filter\!!es,
                                   packagedata.enigma.callbacks[ \!!bs#1\!!es],
                                   \!!bs#1\!!es)
      else
        print\!!bs ENIGMA: No machine of that name: #1!\!!es
        os.exit()
      end
    }%
  }%
  \@EA\gdef\csname stop\current_enigma_id\endcsname{%
    \endgraf
    \directlua{
      luatexbase.remove_from_callback(\!!bs\c!pre_linebreak_filter\!!es, \!!bs#1\!!es)
      packagedata.enigma.machines[ \!!bs#1\!!es]:processed_chars()
    }%
    \egroup%
  }%
}

\def\defineenigma#1{%
  \begingroup
  \let\@EA\expandafter
  \edef\current_enigma_id{#1}%
  \@EA\do_define_enigma\@EA{\current_enigma_id}%
  \endgroup%
}

%D \stopdocsection

\catcode`\_=8  % \popcatcodes
\catcode`\!=12 % reserved according to source2e
\ifenigmaisrunningplain\catcode`\@=12\fi
% vim:ft=tex:sw=2:ts=2:expandtab:tw=72
