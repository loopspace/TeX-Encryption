\startdocchapter[title=Usage]

\startdocsection[title=Loading the Module/Package]
  \TODO{instuctions for plain, latex + ctx}
\stopdocsection

\startdocsection[title=Options Explained]
  \TODO{day key syntax, rotor settings strings etc.}

%%% other_chars
Most documents don’t naturally adhere to the machine-imposed restriction
to the 26 letters of the Latin alphabet. The original encipherment
directives comprised substitution tables to compensate for a set of intrinsic
peculiarities of the German language, like umlauts and common digraphs.
The \emph{Enigma} simulation module strives to apply these automatically
but there is no guarantee of completeness.

However, the Enigma lacks means of handling languages other than German.
When the substitution lookup fails, there are two ways of proceeding:
either to ignore the current character or to pass it on to the output as
if nothing happened. The default behaviour is to drop alien letters and
move on. If the user intends to keep these foreign characters instead,
E can achieve this by setting the \identifier{other_chars} key in the
Enigma setup to the value \emph{true}. An example of how the result of
both methods may look, other thing being equal, is given in below
listing (example for \CONTEXT).

\startcontexttyping
\usemodule [enigma]
\defineenigma [secretmessage]
\setupenigma  [secretmessage] [
  other_chars = yes,
  day_key = B V III II 12 03 01 GI JV KZ WM PU QY AD CN ET FL,
  rotor_setting = ben,
]

\defineenigma [othermessage] [secretmessage]
\setupenigma  [othermessage] [other_chars=wrong]

\starttext

\startsecretmessage
  føo bąr baž
\stopsecretmessage
\startothermessage
  føo bąr baž
\stopothermessage

\stoptext
\stopcontexttyping

Both methods have their disadvantages: if the user chooses to have the
unknown characters removed it might distort the decrypted text to
becoming illegible. Far more serious, however, are the consequences of
keeping them. As artefacts in the ciphertext they would convey
information about the structure of the plain text.

\stopdocsection

\startdocsection[title=Basic Functionality]

Encrypt the text of your document using the script interface. For
a start try out the settings as given in below listing.

\starttyping
mtxrun --script mtx-t-enigma                    \
       --setup="day_key = B I II III 01 01 01,  \
                rotor_setting = xyz,            \
                verbose=0"                      \
       --text="Gentlemen don’t read each other’s mail, Mr. Turing\!"
\stoptyping

This will result in the thoroughly scrambled string
\type{omribshpwfrfjovkntgqgiabbkhjpxmhdztapkatwrvf}.
Then, use the same settings you encrypted the text with in your
document.

\startcontexttyping
\usemodule[enigma]
\defineenigma [secretmessage]
\setupenigma  [secretmessage] [
  day_key = B I II III 01 01 01,
  rotor_setting = xyz,
  verbose=3,
]

\starttext

\startsecretmessage
  omribshpwfrfjovkntgqgiabbkhjpxmhdztapkatwrvf
\stopsecretmessage

\stoptext
\stopcontexttyping

If you compile this document with \CONTEXT, the plain text will
reappear.
Notice that punctuation is substituted with the letter “x” before
encryption and that spaces are omitted.

\useURL[kgbuch] [http://de.wikipedia.org/wiki/Kenngruppenbuch]
       []       [code book]
Now it’s certainly not wise to carry around the key to encrypted
documents as plain text within those documents. The keys will have to be
distributed via an independent channel, e.\,g. a \from[kgbuch].
Keys in general don’t have to be supplied inside the document. If there
is none specified, the module will interrupt the \TEX\ run and
\emph{ask} for user input. Suppose Alice wanted to send an encrypted
file to Bob and already generated the cipher text as follows:

\starttyping
mtxrun --script mtx-t-enigma                                    \
        --setup="day_key =B I IV V 22 07 10 AZ DG IE YJ QM CW,  \
                 rotor_setting = bar,                           \
                 verbose=0"                                     \
        --text="I have nothing to hide. From the NSA, that is."
\stoptyping

Alice would then include the result of this line in her \LATEX\ document
as follows:

\startlatextyping
\documentclass{scrartcl}
\usepackage{enigma}
\defineenigma{decryption}
%% Encryption key not given in the setup.
\setupenigma{decryption}{
  rotor_setting = bar,
  verbose       = 3,
}
\begin{document}

\startdecryption
usbatbwcaajhzgeyzkqskupzbmdhbdepccgeh
\stopdecryption

\end{document}
\stoplatextyping

She subsequently mails this file to Bob and conveys the key through a
secure channel. They only thing that will be left for Bob to do now, is
to enter the key at the prompt when compiling the document with
\LUALATEX.
\stopdocsection

\startdocsection[title=Uses and Abuses]

In \LUATEX, \identifier{callback}s may stack. This allows filtering the
input through many enigma machines successively. For instance, in the
following listing, two instances of the same machine are generated and
applied.

\startcontexttyping
\usemodule[enigma]              %% load the module
\defineenigma [secretmessage]   %% generate and
\setupenigma  [secretmessage] [ %% configure a machine
  day_key = B IV V II 01 01 01 AD CN ET FL GI JV KZ PU QY WX,
  rotor_setting = foo,
  verbose=3,
]

%% now, copy the first machine’s settings
\defineenigma [othermessage] [secretmessage]

%% here we go!
\starttext

\startothermessage  %% enable machine 1
\startsecretmessage %% enable machine 2 while no 1 is active
Encryption equals decryption.
\stopothermessage
\stopsecretmessage

\stoptext \endinput
\stopcontexttyping

\stopdocsection

\stopdocchapter

\startdocchapter[title=Metadata]
\startdocsection[title=License]

© 2012 \emph{Philipp Gesang}. All rights reserved.

Redistribution and use in source and binary forms, with or
without modification, are permitted provided that the following
conditions are met:

\startitemize[n]
  \item Redistributions of source code must retain the above
    copyright notice, this list of conditions and the following
    disclaimer.
  \item Redistributions in binary form must reproduce the
    above copyright notice, this list of conditions and the
    following disclaimer in the documentation and/or other
    materials provided with the distribution.
\stopitemize

\begingroup
\setuptolerance [horizontal,strict]
\startalignment [right,nothyphenated]
\noindentation\startsmallcaps
  this software is provided by the copyright
  holder “as is” and any express or implied warranties,
  including, but not limited to, the implied warranties of
  merchantability and fitness for a particular purpose are
  disclaimed. in no event shall the copyright holder or
  contributors be liable for any direct, indirect, incidental,
  special, exemplary, or consequential damages (including, but
  not limited to, procurement of substitute goods or services;
  loss of use, data, or profits; or business interruption)
  however caused and on any theory of liability, whether in
  contract, strict liability, or tort (including negligence or
  otherwise) arising in any way out of the use of this software,
  even if advised of the possibility of such damage.
\stopsmallcaps\endgraf
\stopalignment
\endgroup
\stopdocsection

\startdocsection[title=Acknowledgements]
The idea to implement the \emph{Enigma} cipher for \TEX\ came up while I
was reading \emph{The Code Book} by Simon Singh.
This work contains an excellent portrayal of the history of German
military cryptography and Allied cryptanalysis before and during the
Second World War.
Also, the Enigma module drew lots of inspiration from Arno Trautmann’s
\identifier{chickenize} package, which remains the unsurpassed hands-on
introduction to callback trickery.
Finally, without \LUATEX\ encryption on node-level would not have been
possible.
\stopdocsection
\stopdocchapter
